\documentclass[RNAAS]{aastex631}
\usepackage{graphicx}

\begin{document}

\title{Characterization of TIC~22888126 as an Active Dwarf Nova Candidate with Possible Ultra-Short Period}

\correspondingauthor{Landon Mutch}
\email{landonmutch@protonmail.com}

\author[0009-0003-3859-9565]{Landon Mutch}
\affiliation{Independent Researcher}

\begin{abstract}
We characterize TIC~22888126 (Gaia DR3 5947829831449229312) as an active dwarf nova candidate based on TESS photometry revealing seven distinct outbursts across 6~years (Sectors 13, 39, 66, 93). Outburst amplitudes range from 1--5~mag with classic dwarf nova morphology. The Gaia DR3 short-timescale variability catalog reports a photometric period of 57.3~minutes; if this represents the orbital period, TIC~22888126 would join V485~Cen and $\sim$10 other CVs known below the 75-minute period minimum. Spectroscopic confirmation of the orbital period is required.
\end{abstract}

\keywords{cataclysmic variables --- novae, dwarf --- methods: statistical --- catalogs}

\textbf{Introduction.} The International Variable Star Index \citep[VSX;][]{Watson2006} lists TIC~22888126 as a generic variable (``VAR'') with photometric period $P=57.3$~min derived from Gaia DR3 short-timescale variability analysis \citep{Gaia2023}. We identified this object through Isolation Forest anomaly detection \citep{Liu2008} applied to Gaia DR3 \texttt{vari\_summary} statistics, where it showed extreme negative skewness ($-3.94$) and high kurtosis ($+18.5$)---signatures consistent with outbursting behavior.

\textbf{Observations.} TIC~22888126 ($\alpha=17^{\rm h}55^{\rm m}28\fs37$, $\delta=-47\degr35\arcmin34\farcs1$, $G=16.58$) is detected in the ROSAT All-Sky Survey but absent from SIMBAD and cataclysmic variable catalogs. We extracted TESS Full Frame Image photometry from Sectors 13, 39, 66, and 93 (spanning 2019--2025) using $3\times3$ pixel apertures without background subtraction to preserve outburst flux.

\textbf{TESS Outbursts.} We detect seven distinct outbursts across 6~years of TESS coverage (Figure~\ref{fig:tess}). The outbursts exhibit classic dwarf nova morphology---rapid rise with exponential decline \citep{Osaki1996}. Sector~13 shows one outburst ($-2.5$~mag); Sector~39 one event ($-1.0$~mag); Sector~66 two outbursts ($-2.5$ and $-3.0$~mag); and Sector~93 three events including two large-amplitude outbursts ($-5.3$ and $-4.9$~mag). Amplitudes ranging from 1--5~mag suggest a mix of normal outbursts and superoutbursts characteristic of SU~UMa-type systems. The presence of multiple outbursts within single 27-day sectors indicates a recurrence time of $\sim$2~weeks during active states.

\begin{figure}[h!]
\centering
\includegraphics[width=\columnwidth]{tic22888126_publication_log.png}
\caption{TESS light curve of TIC~22888126 (log scale) showing seven outbursts across Sectors 13, 39, 66, and 93 (2019--2025). Red points indicate flux $>2\times$ quiescent. The 5-mag events in Sector 93 suggest superoutbursts.}
\label{fig:tess}
\end{figure}

\textbf{Classification and Period.} The frequent outbursts, variable amplitude (1--5~mag), X-ray detection, and outburst morphology strongly support dwarf nova classification. The Gaia-derived 57.3-min photometric period, if confirmed as the orbital period, would place TIC~22888126 among $\sim$12 known CVs below the 75-minute period minimum, comparable to V485~Cen \citep[$P_{\rm orb}=59$~min;][]{Augusteijn1996}. Such short-period systems have evolved past the CV period minimum and likely host degenerate or semi-degenerate donors \citep{Knigge2011}. However, we caution that the Gaia period is photometric and could represent a superhump period (typically 1--5\% longer than orbital in SU~UMa stars) or other periodic signal. Time-resolved spectroscopy is essential to confirm the orbital period.

\textbf{Conclusions.} TIC~22888126 is a strong dwarf nova candidate showing frequent outbursts with SU~UMa-like behavior. If the 57.3-min photometric period is confirmed as orbital, this would be a rare, highly-evolved system valuable for testing CV evolution models. We recommend spectroscopic observations to confirm the cataclysmic variable nature, measure the true orbital period, and determine whether the large-amplitude events are superoutbursts.

\textbf{Data Availability.} TESS data: MAST \dataset[10.17909/t3jq-kp18]{https://doi.org/10.17909/t3jq-kp18}. Analysis code: \url{https://github.com/toadlyBroodle/science}

\begin{acknowledgments}
This work used data from ESA Gaia (processed by DPAC) and NASA TESS (via MAST/STScI). This research used ROSAT, AAVSO VSX, and VizieR (CDS). Computing: Google Colab. Analysis: Anthropic Claude.
\end{acknowledgments}

\facilities{Gaia, TESS, ROSAT}

\software{astropy \citep{Astropy2022}, scikit-learn \citep{Pedregosa2011}, astroquery}

\begin{thebibliography}{12}
\bibitem[Astropy Collaboration(2022)]{Astropy2022} Astropy Collaboration 2022, \apj, 935, 167
\bibitem[Augusteijn et al.(1996)]{Augusteijn1996} Augusteijn, T., van der Hooft, F., de Jong, J.~A., \& van Paradijs, J.\ 1996, \aap, 311, 889
\bibitem[Gaia Collaboration(2023)]{Gaia2023} Gaia Collaboration 2023, \aap, 674, A1
\bibitem[Knigge et al.(2011)]{Knigge2011} Knigge, C., Baraffe, I., \& Patterson, J.\ 2011, \apjs, 194, 28
\bibitem[Liu et al.(2008)]{Liu2008} Liu, F.~T., Ting, K.~M., \& Zhou, Z.-H.\ 2008, Proc.\ ICDM, 413
\bibitem[Osaki(1996)]{Osaki1996} Osaki, Y.\ 1996, \pasp, 108, 39
\bibitem[Pedregosa et al.(2011)]{Pedregosa2011} Pedregosa, F.\ et al.\ 2011, JMLR, 12, 2825
\bibitem[Watson et al.(2006)]{Watson2006} Watson, C.~L.\ et al.\ 2006, SASS, 25, 47
\end{thebibliography}

\end{document}
